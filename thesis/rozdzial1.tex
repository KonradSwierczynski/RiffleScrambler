\chapter{Analiza problemu}
\thispagestyle{chapterBeginStyle}
\label{rozdzial1}

\begin{definition}
	(\textit{Parallel/Sequential Graph Pebbling}). Niech $G = (V, E)$ będzie grafem skierowanym grafem acyklicznym i niech $T \subset V$ będzie podzbiorem wierzchołków do oetykietowania, nazywanym celem.
	Stanem etykietowania $G$ jest podzbiór $P_{i} \subset V$.
	Poprawnym etykietowaniem równoległym jest ciąg $P = (P_{0}, \dots , P_{t})$ stanów etykietowania $G$,
	gdzie $P_{0} = \emptyset $ oraz gdzie spełnione są warunki 1 oraz 2 poniżej.
	Etykietowanie sekwencyjne musi dodatkowo spełniać warunek 3.
	\begin{enumerate}
		\item Każdy wierzchołek z celu jest w pewnej konfigutracji oetykietowany (nie koniecznie wszytkie jednocześnie).
		$$ \forall x \in T \exists x \leq t : x \in P_{x} $$
		
		\item Oetykietować wierzchołek można tylko wtedy, gdy wyszyscy jego rodzice
		są oetykietowani w poprzenim kroku.
		$$ \forall i \in [t] : x \in (P_{i} \setminus P_{i-1}) \Rightarrow parents(x) \subset P_{i-1} $$
		
		\item W każdym kroku można oetykietować co najwyżej jeden wierzchołek.
		$$ \forall i \in [t]: | P_{i} \setminus P_{i-1} | \leq 1 $$
	\end{enumerate}
\end{definition}


\begin{proof}
	dowód trywialny
\end{proof}

