\chapter{Wstęp}
\thispagestyle{chapterBeginStyle}


Niniejsza praca swoim zakresem obejmuje kryptograficzne funkcje do przechowywania haseł typu \textit{memory-hard} zapewniającą wysokie bezpieczeństwo przechowywania haseł.
Celem pracy jest analiza bezpieczeństwa oraz implementacja funkcji tego typu, jaką jest RiffleScrambler \cite{rs}. Praca przedstawia także definicje opisu jakości takich funkcji oraz przedstawia porównanie RiffleScrambler do obecnych rozwiązań.

W rozdziale \ref{rozdzial1} przedstawiono motywację, formalne definicje oraz twierdzenia potrzebne po przeprowadzenia analizy bezpieczeństwa oraz porównywania funkcji \textit{memory-hard}.
Rozdział \ref{razdzial2} przedstawia opis algorytmów, analizę bezpieczeństwa i porównanie do istniejących rozwiązań.
Rozdział \ref{rozdzial3} opisuje wykonane prace implementacyjne oraz dokumentację interfejsu.
W rozdziale \ref{rozdzial4} przedstawiono wymagania oraz sposób uruchomienia implementacji.
\nameref{podsumowanie} zawiera podsumowanie ukończonych prac oraz uzyskanych wyników.

