\chapter{Implementacja}
\thispagestyle{chapterBeginStyle}

Implementacja $\mathbf{RiffleScrambler}$ została napisana tak, aby spełniała następujące założenia:
\begin{itemize}
	\item spełnianie wymagań konkursu na funkcję do przechowywania haseł \cite[PHC, ang \textit{Password Hashing Competition}]{PHC2013}, który rozgrywał się w latach 2013 - 2015,
	
	\item zapewnienie najwyższej wydajności obliczania, aby można było stosować parametry zapewniające wysokie bezpieczeństwo w akceptowalnym czasie obliczeń.
\end{itemize}

\section{Wymagania PHC}
Wymagania konkursy PHC odnoszące się do implementacji:
\begin{itemize}
	\item implementacja powinna być napisana w języku C(++) w taki sposób, aby była przenośna,
	\item do implementacji powinny być dołączone instrukcje kompilacji (np. Makefile),
	\item interfejs powinien być dostępny do użycia w języku C, oraz powinien zapewniać funkcję z podaną sygnaturą,
	\item implementacja może używać biblioteki OpenSSL,
	\item do implementacji powinny zostać dołączone testy.
\end{itemize}

% Udostępnia środowisko do pisania kodu
% \usepackage{minted}

% \begin{minted}{C}
% int PHS(void *out, size_t outlen, const void *in, size_t inlen, const void *salt, size_t saltlen, % unsigned int t_cost, unsigned int m_cost); 
% \end{minted}


\section{Opis technologii}
Początkowa implementacja napisana została w języku Python 3.7, jednak ze względu na niską wydajność  napisana została ostatecznie w języku C++17, standard ISO/IEC 14882:2017.
Dzięki temu wydajność wzrosła ponad dziesięciokrotnie.
W implementacji używana jest biblioteka OpenSSL w wersji ...,
Z podanej biblioteki użyto interfejsu EVP który dostarcza podstawowe kryptograficzne funkcje skrótu takie jak SHA2, SHA3, blake2s czy ripemd160.


\section{Omówienie kodów źródłowych}
Omówiony zostanie interfejs funkcji $\mathbf{RiffleScrambler}$, który 

\section{Parametry}


\section{Wydajność}


\section{Przykład użycia}


